\documentclass[12pt]{beamer}
\usepackage[utf8]{inputenc}
\usepackage[T1]{fontenc}
\usepackage[portuguese]{babel}
\usepackage{graphicx}
\usepackage{color}
\usepackage{slidesdef}

\title{Git - Trabalhando com repositórios}
\author{Laboratório de Pesquisa Operacional e Otimização \\
  Abel Siqueira \\
  Raniere Silva
}

\date{ 27 de Setembro de 2014 }

% Use com um terminal flututante

\newcommand{\cmd}[1]{
\begin{flushleft}
{\color{yellow} \tt \$ #1 }\\
\end{flushleft}
}

\newcommand{\cmdinline}[1]{
{\color{yellow} \tt #1} }

\newcommand{\cmmt}[1]{
{\color{magenta} \tt \# #1} }

\newcommand{\bashgt}{ \textgreater\ }
\newcommand{\ddash}{-{}-}

\begin{document}

\myframe{
  \maketitle
}

\tikzstyle{local}=[draw, rectangle]
\tikzstyle{remoto}=[draw, circle]

\myframe{
  \begin{center}
    \begin{tikzpicture}[thick, node distance=3cm, bend angle=20, shorten <=1pt,
      shorten >=1pt, bend right]
      \node[draw, rectangle] (voce)   {Você};
      \onslide<2->{
        \node[draw, circle] (remote) [above right of=voce] {Remoto};
        \draw[->] (voce) to (remote);
      }
      \onslide<3->{
        \draw[->] (remote) to (voce);
      }
      \onslide<4->{
        \node[draw, rectangle] (note) [below right of=remote] {Notebook};
        \draw[->] (note) to (remote);
        \draw[->] (remote) to (note);
      }
      \onslide<5->{
        \node[draw, circle] (fork) [above of=remote] {Fork};
        \draw[->] (remote) to (fork);
      }
      \onslide<6->{
        \draw[->] (fork) to (remote);
        \draw[->] (fork) to (voce);
        \draw[->, bend left] (fork) to (note);
      }

    \end{tikzpicture}
  \end{center}
}

\myframe{
  \ctr{Repositórios online}
  \begin{itemize}
    \item \only<1>{Github}\only<2>{{\color{red}\bf Github}}
    \item Bitbucket
    \item Gitorious
  \end{itemize}
}

\myframe{
  \ctr{\Huge Fim}
}

\end{document}
