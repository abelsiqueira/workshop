\documentclass[12pt]{article}
\usepackage[utf8]{inputenc}
\usepackage[T1]{fontenc}
\usepackage{nopageno}
\usepackage{color}

\usepackage[top=1cm,bottom=2cm,left=1cm,right=1cm, paperwidth=16cm,
paperheight=16cm]{geometry}
\title{Shell - Bash}
\author{Laboratório de Pesquisa Operacional e Otimização \\
Abel Siqueira \\
Raniere Silva
}
\date{ 21 de Fevereiro de 2014 }

\newcommand{\aframe}[1]{
\Large
\begin{center}
#1
\end{center}
\vfill
\newpage
}
\newcommand{\tit}[1]{
{\LARGE \color{green} #1}
\\ \vspace{0.5cm}
\vfill
}
\newcommand{\cmd}[1]{
\begin{flushleft}
{\color{yellow} \tt \$ #1 }\\
\end{flushleft}
}
\newcommand{\cmdinline}[1]{
{\color{yellow} \tt #1} }

\pagecolor{black}
\color{white}

\begin{document}

\maketitle
\newpage

\aframe{
\tit{Estrutura de arquivos}

\begin{description}
\item[/] \hfill
\begin{description}
  \item[/bin] \hfill
  \item[/home] \hfill
  \begin{description}
    \item[/home/abel] \hfill
  \end{description}
  \item[/lib] \hfill
  \item[/media] \hfill
  \item[/mnt] \hfill
  \item[/tmp] \hfill
\end{description}
\end{description}
}

\aframe{
\tit{Explorando}
Nenhum destes comandos quebra o computador:
\cmd{ls}
\cmd{ls / -F}
\cmd{ls /home}
\cmd{ls -la}
\cmd{pwd}
\cmd{cd /home}
\cmd{cd ..}
\cmd{cd /tmp}
\cmd{mkdir novodir}
\cmd{ls \&\& cd novodir}
}

\aframe{
\tit{Quero mais}
Será que dá pra listar por tamanho?
\begin{itemize}
  \item Google
  \item {\tt man}
\end{itemize}
\cmd{man}
Use \cmdinline{q} para sair do \cmdinline{man}.
Procurando por size, encontramos a opção \cmdinline{-s}.
\cmd{ls -S}
\cmd{ls -Sl}
}

\aframe{
\tit{Manipulação de arquivos}
Crie um arquivo qualquer. Vamos chamá-lo de 
\cmdinline{arquivo.txt}. Dentro dele escreva seu nome.
Salve na pasta que criamos anteriormente.
\cmd{ls}
\cmd{cat arquivo.txt}
\cmd{cp arquivo.txt meunome.txt}
\cmd{cat meunome.txt}
\cmd{mv arquivo.txt outro.txt}
\cmd{ls}
\cmd{rm outro.txt}
}

\aframe{
\tit{Tão longe de casa}
\cmd{pwd}
\cmd{cd /home/abel}
\cmd{cd /usr/local/share}
\cmd{cd \textasciitilde}
\cmd{cd /usr/local/share}
\cmd{cd}
\cmd{echo "Hello"}
\cmd{echo ~}
\cmd{echo \$HOME}
}

\aframe{
\tit{Redirecionando}
\cmd{cd /tmp}
\cmd{echo "Abel" > testing}
\cmd{cat testing}
\cmd{python2 -c "s=raw\_input(); print s"}
\cmd{python2 -c "s=raw\_input(); print s" < tst}
\cmd{echo "Zé" | python2 -c "s=raw\_input(); print s"}
}

\aframe{
\tit{Exercícios}
\begin{enumerate}
  \item Crie uma pasta chamada \cmdinline{exshell} em seu 
    diretório pessoal.
  \item Crie um arquivo chamado \cmdinline{execs} nessa pasta 
    contendo toda a listagem da pasta \cmdinline{/bin}.
  \item O comando \cmdinline{tr a b} troca a letra \cmdinline{a} pela
    letra \cmdinline{b} de algum texto redirecionado para o comando.
    Redirecione o arquivo \cmdinline{execs} para o comando \cmdinline{tr} para
    trocar a letra \cmdinline{z} pela letra \cmdinline{a}. (Existem duas
    maneiras de fazer).
  \item A saída acima ficou fora de ordem. Use o comando \cmdinline{sort} para
    reodernar as linhas sem criar arquivos adicionais.
\end{enumerate}
}

\end{document}
