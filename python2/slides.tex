\documentclass[12pt]{article}
\usepackage[utf8]{inputenc}
\usepackage[T1]{fontenc}
\usepackage{nopageno}
\usepackage{listings}
\usepackage[usenames,dvipsnames]{color}

\usepackage[top=1cm,bottom=2cm,left=1cm,right=1cm, paperwidth=16cm,
paperheight=16cm]{geometry}
\title{Python2}
\author{Laboratório de Pesquisa Operacional e Otimização \\
Abel Siqueira \\
Raniere Silva
}
\date{ 21 de Fevereiro de 2014 }

\newcommand{\iframe}[1]{
\Large
\begin{center}
#1
\end{center}
}
\newcommand{\iframeend}{ \vfill \newpage}
\newcommand{\aframe}[1]{
\Large
\begin{center}
#1
\end{center}
\vfill
\newpage
}

\newcommand{\tit}[1]{
{\LARGE \color{green} #1}
\\ \vspace{0.5cm}
\vfill
}
\newcommand{\cmd}[1]{
\begin{flushleft}
{\color{yellow} \tt \$ #1 }\\
\end{flushleft}
}
\newcommand{\cmdpy}[1]{
\begin{flushleft}
{\color{yellow} \tt > #1 }\\
\end{flushleft}
}
\newcommand{\inline}[1]{
{\color{yellow} \tt #1} }

\definecolor{filebg}{gray}{0.1}
\lstdefinestyle{filestyle}{
  language=Python,
  basicstyle=\ttfamily,
  keywordstyle=\color{red},
  commentstyle=\itshape\color{cyan},
  stringstyle=\color{green},
  frame=single,
  breaklines=true,
  breakatwhitespace=false,
  backgroundcolor=\color{filebg}
}
\lstnewenvironment{file}{\lstset{style=filestyle}}{}

\pagecolor{black}
\color{white}

\begin{document}

\maketitle
\newpage

\aframe{
  \tit{Introdução}
  Podemos criar arquivos e rodá-los com \inline{python2}, 
  ou utilizar o Interpretador Interativo.

  Vamos começar usando o interpretador. Para isso, digite
  \cmd{python2}
  Obtendo
  \cmdpy{}
}

\aframe{
\tit{O famoso... }
\cmdpy{print "Hello World" \\
Hello World}
}

\aframe{
  \tit{Exemplos básicos}
  \cmdpy{2 + 3}
  \cmdpy{10*20}
  \cmdpy{3**5}
  \cmdpy{2\textasciicircum 1 \#Não é potência}
}

\iframe{
\tit{Arquivo básico}
Crie um arquivo \inline{test1.py} com o conteúdo
}
\begin{file}
s = raw_input()
print s.swapcase()
\end{file}
\iframe{
Execute com
\cmd{python2 test1.py}
e digite qualquer coisa.
}
\iframeend

\iframe{
  \tit{Algo útil}
  Vamos criar um programa para fatorial.
  Para isso, usaremos funções.

  Crie o arquivo \inline{fact.py}.
}
\begin{file}
def factorial(n):
  # Calcula o fatorial
  return 1
\end{file}
\iframe{
  No interpretador faça
  \cmdpy{from fact import factorial}
  \cmdpy{factorial(1) \\
  1}
  \cmdpy{factorial(5) \\
  1}
}
\iframeend

\iframe{
  \tit{Algo útil}
  Melhorando
}
\begin{file}
def factorial(n):
  # Calcula o fatorial
  if n <= 0:
    return 1
  else
    return n*factorial(n-1)
\end{file}
\iframeend

\iframe{
  \tit{Algo útil}
  E também, para não ter que chamar sempre o interpretador,
  crie o arquivo \inline{callfact.py}.
}
\begin{file}
from fact import factorial

print factorial(0)
print factorial(2)
print factorial(5)
print factorial(-1)
\end{file}
\iframe{
  E chamamos
  \cmd{python2 callfact.py \\
1 \\
2 \\
120 \\
1
}
}
\iframeend
\iframe{
  \tit{Algo útil}
  Vamos fazer com que o fatorial de um número negativo retorne 0.
}
\begin{file}
def factorial(n):
  # Calcula o fatorial
  if n < 0:
    return 0
  elif n <= 1:
    return 1
  else
    return n*factorial(n-1)
\end{file}
\iframeend

\aframe{
  \tit{Algo útil}
  Rodando novamente, obtemos
  \cmd{python2 callfact.py \\
1 \\
2 \\
120 \\
0
}
}

\end{document}
